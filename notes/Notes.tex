\documentclass[table]{article}
\usepackage[utf8]{inputenc}
\usepackage[T1]{fontenc}
\usepackage{xcolor}



\title{Notes Projet Typescript}
\author{SALL Ibrahima}
\date{Mai 2021}




\begin{document}

\maketitle{\begin{center}\textbf{Notes} \end{center}}



Jest: Install globally with \textbf{npm install -g jest} to be able to command jest --init.
photo

ne pas oublier:     npm install ts-node --save-dev
npm install crypto-js -- fonction de hachage qui nous permettra de coder du SHA256
npm run lint
npm run test 
npm run start
npm run test -- --watch
\newline Il est conseillé si on a le choix d'éviter les itérations sur les objet.
\newline TypeScript : On peut installer TypeScript après avoir installé Node.js en exécutant la commande suivante dans un terminal ou une fenêtre Commande : \textbf{npm install -g typescript}.
\newline Start node project: Use \textbf{npm init} 
This utility will walk you through creating a package.json file.
It only covers the most common items, and tries to guess sensible defaults.

See `npm help init` for definitive documentation on these fields
and exactly what they do.

Use `npm install <pkg>` afterwards to install a package and
save it as a dependency in the package.json file. 
\newline \textbf{npm install -D typescript@3.3.3} npm notice created a lockfile as package-lock.json.
\newline \textbf{npm install -D tslint@5.12.}.

\newline Ce paquet est nécessaire car TypeScript et Express sont des paquets indépendants. Sans le paquet @types/express, il n'y a aucun moyen pour TypeScript de connaître les types de classes Express.
\newline Une alternative à la création et au remplissage manuels du fichier tsconfig.json consiste à exécuter la commande suivante :
\textbf{tsc --init} Cette commande va générer un fichier tsconfig.json bien commenté.
\newline Pour configurer le linting TypeScript pour le projet. Il faut ans un terminal fonctionnant au root du répertoire du projet, que ce tutoriel a établi comme node_project, exécutez la commande suivante pour générer un fichier \textbf{tslint.json} :
\textbf{./node_modules/.bin/tslint --init}
\newline tslint standard file content: /*{
    "defaultSeverity": "error",
    "extends": [
        "tslint:recommended"
    ],
    "jsRules": {},
    "rules": {"no-console": false},
    "rulesDirectory": []
}*/
\newline - run: npm ci
- run: npm run build --if-present


\end{document}