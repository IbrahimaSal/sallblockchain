\documentclass[table]{article}
\usepackage[utf8]{inputenc}
\usepackage[T1]{fontenc}
\usepackage{xcolor}



\title{Notes Projet Typescript}
\author{SALL Ibrahima}
\date{Mai 2021}




\begin{document}

\maketitle{\begin{center}\textbf{Notes} \end{center}}



Jest: Install globally with \textbf{npm install -g jest} to be able to command jest --init.
photo

ne pas oublier:     npm install ts-node --save-dev
npm install crypto-js -- fonction de hachage qui nous permettra de coder du SHA256
npm run lint
npm run test 
npm run start
npm run test -- --watch
\newline Il est conseillé si on a le choix d'éviter les itérations sur les objet.
\newline TypeScript : On peut installer TypeScript après avoir installé Node.js en exécutant la commande suivante dans un terminal ou une fenêtre Commande : \textbf{npm install -g typescript}.
\newline Start node project: Use \textbf{npm init} 
This utility will walk you through creating a package.json file.
It only covers the most common items, and tries to guess sensible defaults.

See `npm help init` for definitive documentation on these fields
and exactly what they do.

Use `npm install <pkg>` afterwards to install a package and
save it as a dependency in the package.json file. 
\newline \textbf{npm install -D typescript@3.3.3} npm notice created a lockfile as package-lock.json.
\newline \textbf{npm install -D tslint@5.12.}.

\end{document}